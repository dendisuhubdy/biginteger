%%%%%%%%%%%%%%%%%%%%%%%%%%%%%%%%%%%%%%%%%%%%%%%%%%%%%%
% Filename:      preliminaries.tex
% Author:        Junwei Wang(wakemecn@gmail.com)
% Last Modified: 2013-12-30 21:17
% Description:
%%%%%%%%%%%%%%%%%%%%%%%%%%%%%%%%%%%%%%%%%%%%%%%%%%%%%%
\section{Preliminaries}

\subsection{Elliptic Curves}
The range of hash function we talk about in this report is the set of points on
specified elliptic
curve, and the following gives the definition of the type of elliptic curves we
adopt in this report.
\begin{defn}[\textbf{Elliptic Curve} \cite{koblitz1987elliptic}]
An elliptic curve $E(\mathbb{F})$ defined over a field $\mathbb{F}$ of characteristic $\neq 2, 3$ 
is the set of solutions $(x, y) \in \mathbb{F}^2$ to the equation:
\begin{equation}
y^2=x^3+ax+b, \ \ \ a, b \in \mathbb{F}
\end{equation}
where the cubic on the right have no multiple roots.
\end{defn}

As it is illustrated in \cite{koblitz1987elliptic}, the points on $E(\mathbb{F})$
together with ``point at infinity" $\mathcal{O}$ can define a group structure,
where $\mathcal{O}$ is the identity element.
Let $P = (x_1, y_1), Q = (x_2, y_2) \in E_\mathbb{F}$, the other group law follows:
\begin{enumerate}
\item $P \neq \mathcal{O}$, then $-P = (x_1, -y_1)$ is denoted
as the \emph{negative} of $P$;
\item $P, Q \neq \mathcal{O}$, then $P+Q=(x_3, y_3)$ with,
\begin{equation}
\label{groupPlus}
\begin{array}{lll}
x_3 & = & \lambda^2-x_1-x_2, \\
y_3 & = & \lambda(x_1-x_3) - y_1, \\
\lambda & = &
\left\{
\begin{array}{ll}
\frac{y_2-y_1}{x_2-x_1} & \mbox{if\ } P \pm Q,\\
\frac{3x_1^2+a}{2y_1}   & \mbox{if\ } P = Q.
\end{array}
\right.
\end{array}
\end{equation}
\end{enumerate}
\par In our implementation, we use a $nG$ to represent ``point at
infinity" $\mathcal{O}$, where $n$ is the total number of points in this group
and $G$ is the generator of the group, since that we cannot represent it in the tradition
two-dimensions coordinates.

\subsection{Quadratic Residue and Legendre Symbol}
We recall the definition of quadratic residue and Legendre Symbol in the
elementary number theory to help to understand the simplified-SWU
algorithm.
\begin{defn}[Quadratic Residue \cite{shoup2009computational}]
Let $p$ be a prime number, an integer $a$ is called a 
\textbf{quadratic residues} modulo $p$, or a square modulo $p$, if 
$a = b^2\ \texttt{mod}\ p$ for some integer $b \in \mathbb{Z}_p^*$, and $b$
is called a \textbf{square root} of a modulo $p$.
\end{defn}
\begin{defn}[Legendre Symbol \cite{shoup2009computational}]
For an odd prime $p$ and an integer $a$:
\begin{equation}
\left(\frac{a}{p}\right) = \left\{
\begin{array}{rl}
1 & \mathrm{if\ } gcd(a, p) \neq 0 \mathrm{\ and\ } a
\mathrm{\ is\ a\ square\ modulo\ } p,\\
0 & \mathrm{if\ } gcd(a, p) = p,\\
-1 & \mathrm{otherwise}.
\end{array}
\right.
\end{equation}
\end{defn}

\begin{theorem}[\cite{shoup2009computational}]
\label{square}
Let $p$ is a prime, and let $a\in \mathbb{Z}$,
\begin{equation}
\left(\frac{a}{p}\right) = a^{\frac{p-1}{2}} \mathtt{\ mod\ } p.
\end{equation}
\end{theorem}

\subsection{Simple Shallue-Woestijne-Ulas Algorithm}
Shallue and Woestijne provided the first deterministic method in \cite{SW06}, and
the method was optimized for hyper\-elliptic curves in \cite{Ulas07}, which is
called SWU algorithm. Recently, SWU algorithm was simplified in
\cite{brier2010efficient}. We recall the simplified SWU algorithm in the
following lemma.

\begin{lemma}[Simplified Ulas Maps (Simplified-SWU) \cite{brier2010efficient}]
\label{swu}
Let $\mathbb{F}_q$ be a field and let $g(x) = x^3 + ax + b$,
where $a, b \neq 0$.  Let:
\begin{equation}
\begin{array}{lll}
X_2(t) & = & \frac{-b}{a}\left(1+\frac{1}{t^4-t^2}\right),\\
X_3(t) & = & -t^2X_2(t),\\
U(t) &= & t^3g(X_2(t)),
\end{array}
\end{equation}
then 
\begin{equation}
\label{u}
U(t)^2 = -g(X_2(t))\cdot g(X_3(t)).
\end{equation}
\end{lemma}
This lemma was proved correct in the original paper.  
\renewcommand{\algorithmicrequire}{\textbf{Input:}}
\renewcommand{\algorithmicensure}{\textbf{Output:}}

The simple SWU algorithm is composed by two part: computing the candidate
points (line 1 to 5) and selecting the correct one (line 6 to 10).
\begin{algorithm} 
\caption{Simplified SWU algorithm \cite{brier2010efficient}}
\label{sswu}
  \begin{algorithmic}[1]
    \Require $\mathbb{F}_q$ such that $q \equiv 3\ (\texttt{mod}\ 4)$, parameters
$a$, $b$ and input $t \in \mathbb{F}_q$
    \Ensure $(x, y) \in E_{a,b}(\mathbb{F}_q)$ where $E_{a,b}: y^2 = x^3 + ax +
b$

    \State $\alpha \gets -t^2$
    \State $X_2 \gets \frac{-b}{a}(1+\frac{1}{\alpha^2+\alpha})$
    \State $X_3 \gets \alpha \cdot X_2$
    \State $h_2 \gets (X_2)^3 + a\cdot X_2 + b$
    \State $h_3 \gets (X_3)^3 + a\cdot X_3 + b$
    \If{$h_2$ is a square}
      \State \Return $(X_2, h_2^{(q+1)/4})$
    \Else
      \State \Return $(X_3, h_3^{(q+1)/4})$
    \EndIf
  \end{algorithmic}
\end{algorithm}
\\This algorithm defines a function $f'_{a,b}: \mathbb{F}_{q} \rightarrow
E(\mathbb{F}_q)$, which has been proved to be a weak-encoding into curves, and can
be used in the general construction of hashing into elliptic curves \cite{brier2010efficient}.

\subsection{Simple Password Exponential Key Exchange}
Simple password exponential key exchange was firstly described in
\cite{Jablon:1996:SPA:242896.242897}, which uses Diffie-Hellman key exchange
\cite{DH76} to establish a shared key between Alice and Bob.
\par Suppose Alice and Bob share a secret password $pw$ in previous,
let $H$ is a hash function maps arbitrary string into elliptic curve. To
establish a session-key, Alice and Bob do the followings:
\begin{enumerate}
\item Alice computes and sends $A = a.H(pw)$ to Bob, where $a \leftarrow
\mathbb{Z}_q$,
\item Bob computes and sends $B = b.H(pw)$ to Alice, where $b \leftarrow
\mathbb{Z}_q$,
\item Alice computes $K = a\cdot B = ab \cdot H(pw)$,
\item Bob computes $K = b\cdot A = ab \cdot H(pw)$.
\end{enumerate}
We implement this scheme to give
an example of the usage of simplified-SWU algorithms and operations on the
elliptic curves.
